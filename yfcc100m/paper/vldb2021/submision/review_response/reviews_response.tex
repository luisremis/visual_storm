\documentclass[11pt]{proposalnsf}
\usepackage{times}
\usepackage{mathptmx}
\usepackage{epsfig}
\usepackage{amsmath}
\usepackage{graphicx}
\usepackage{url}
\usepackage{listings}
\usepackage{array}
\usepackage[noend,nofillcomment]{algorithm2e}
\usepackage{verbatim}
%\usepackage{algorithmic}
%\usepackage{amsfonts}
%\usepackage{lineno}
\usepackage{ifthen}
\usepackage{booktabs}

\usepackage{color}

\usepackage{multirow}
\usepackage{balance}  % for  \balance command ON LAST PAGE  (only there!)

\makeatletter
\def\tightitemize{\ifnum \@itemdepth >3 \@toodeep\else \advance\@itemdepth \@ne
\edef\@itemitem{labelitem\romannumeral\the\@itemdepth}%
\list{\csname\@itemitem\endcsname}{\setlength{\topsep}{-\parskip}\setlength{\parsep}{0in}\setlength{\itemsep}{0in}\setlength{\parskip}{0in}\def\makelabel##1{\hss\llap{##1}}}\fi}
\let\endtightitemize =\endlist
\def\tightenumerate{%
\ifnum \@enumdepth >\thr@@\@toodeep\else
\advance\@enumdepth\@ne
\edef\@enumctr{enum\romannumeral\the\@enumdepth}%
\expandafter
\list
\csname label\@enumctr\endcsname
{\setlength{\topsep}{-\parskip}\setlength{\parsep}{0in}\setlength{\itemsep}{0in}\setlength{\parskip}{0in}\usecounter\@enumctr\def\makelabel##1{\hss\llap{##1}}}%
\fi}
\let\endtightenumerate =\endlist
\makeatother
\renewcommand{\refname}{\centerline{References cited}}

% this handles hanging indents for publications
\def\rrr#1\\{\par
\medskip\hbox{\vbox{\parindent=2em\hsize=6.12in
\hangindent=4em\hangafter=1#1}}}

\def\baselinestretch{1}

\begin{document}

\begin{center}
{\Large{\bf VLDB 2021 - Response to Reviews: \\
Using VDMS to Index and Search 100M Images}}

\end{center}

\section{Reviewer \#1}

\textbf{
D1 (W1): Although VDMS mentions that it supports Visual Feature Vectors as
first class citizens in Section 3.3, the queries used in evaluation
do not include an example.
}\bigskip


Not for this work, we will add to future work.
Also, building a comparison baseline that integrates all that functionality
is very ad-hoc, and would complicate the understanding of the origins
of the benefits of VDMS.

We can provide appendix.

\bigskip
\textbf{
D2 (W1): All queries involve “resize” operation, which is expensive for the
baselines since it is happening on the client side.
A simplistic query without resizing can show how the graph model performs
against traditional RDS DBs (i.e., will show that VDMS has comparable
performance even for such use-case without "visual manipulation").
}\bigskip

This is a great point, we very much appreciate the suggestion.
We have run a new experiment and added a new figure to the paper (Figure 9),
where we compare 3 of the queries with and without the resize operation
to understand the difference in performance better.
We have also added an explanation on these differences in performance
in the final version.
Note that, even if the resize happens on the client side in the case of both
baselines we have built, the computing capabilities of the server and the
client are identical. This is on purpose, to avoid the difference in
computing power for the resize operation to affect the understanding
of the results.


\bigskip
\textbf{
D3 (W2): It is not clear which queries are supported or not.
The queries used for evaluation are based on tags and geo search.
What about visual-based queries? For example, is image similarity search based
on Visual Feature Vectors within a geographical region supported?
}\bigskip

VDMS does support the suggested query, i.e., perform a similarity search based
on feature vectors and also filter by geographical location.
This can be achieved using the very similar queries as the one used in
the present evaluation, plus the use of the feature vector queries,
that are natively supported using the same set of APIs.
Because of space constraints, we were not able to accommodate initial evaluation
we have been working on with Feature Vector search specifically,
and we plan to perform a more comprehensive evaluation on this functionality
of this feature because of its relevance when processing visual data, as well
as our own needs in our use cases.

\bigskip
\textbf{
D4 (W2): Section 4.3.1, paragraph 5, mentions that INTERSECTION operation is not
yet implemented at the server side.
A discussion on current limitations/future work for such queries or queries
such as in D3, will be better to be clarified in paper.
}\bigskip

Thanks for this comment, we have added a "Future Work" section where we cover
this and D3, expanding further on the most immediate lines of
improvement and research for the VDMS project.

\bigskip
\textbf{
D5 (W3): Section 4.2 paragraph 1, desing
}\bigskip

Fixed in final version.

\bigskip

\newpage
\section{Reviewer \#4}

\bigskip
\textbf{
D1. The proposed solution seems to use the functionalities of the existing systems
as it is. Authors should provide more original ideas or more innovative designs.
For example, providing optimizations that specifically target visual data or query
patterns associated could have been more meaningful to the research communities and
convince users to replace their customized solution to VDMS.
}\bigskip

We partially address this in D3 and also elaborated more on the benefits of
using VDMS over relational databases in Section 3.1.

\bigskip
\textbf{
D2. Both baselines are similar in architecture, and the evaluation has not effectively
highlighted the performance gain of other design choices authors made in VDMS,
except that the VDMS server process both metadata operation and image operation
within only one round of communication with clients.
For example, benchmarking systems can be built to test in isolation the
improvement on metadata handling.
}\bigskip

Thank you for the input. To demonstrate the benefits of VDMS more clearly,
we have added Figure 9, which compares queries with and without the resize operation.
Without the resize operation, the evaluation demonstrates the
improvement on metadata handling better, plus the effect of a single round
of communication between client and the server (as it is the case with VDMS).

\bigskip
\textbf{
D3. The paper should detail the types of queries, and image preprocessing it supports,
as this being “one of the most differentiating aspects” as mentioned by them.
And for the selection of operations supported, authors need to argue for their
decision in the application contexts.
}\bigskip

Thank you for this recommendation.
We have provided additional details on which operations we support and our
decision for choosing the operations in section 3.

\bigskip
\textbf{
D4. Data scientists commonly perform data exploration on raw data and interactively
curate their datasets for downstream tasks like model training.
VDMS could design new mechanisms to support interactive analysis,
which would be beneficial.
}\bigskip

Interactive analysis is out of scope for this paper but this can easily be done
using the current implementation of VDMS.
To support interactive analysis, a new API/frontend could be implemented on top of
VDMS as a web app without adding any additional abstractions.
An example of such front-end can be found <http://coco.datasets.aperturedata.io>.
The front-end is very thin, as most of the heavy-load is done by VDMS through
the interfaces mentioned in this work
(plus a thin REST abstraction we have added recently).

\bigskip
\textbf{
D5. In terms of scalability, the paper only shows that VDMS can handle large
datasets, yet it is not known how VDMS can scale horizontally with more
computation resources.
For more complex logic of image preprocessing, VDMS will suffer from its
single server capacity.
}\bigskip

Thanks for this comment, we have added a "Future Work" section where the need
for this capability in a distributed version of VDMS.

\newpage
\section{Reviewer \#5}

\bigskip
\textbf{
D1: VDMS is a valuable contribution to Database systems where it takes
some first important steps to handle visual data as a first-class citizen
instead of considering it a blob of information. However, the description
of VDMS is rather short (only 4 pages), and some critical technical
details are missing. In particular, what are the technical challenges
for supporting ACID properties using the graph model abstraction? Are
the solutions analogous to RDBMS’s? Does VDMS also support crash
recovery like an RDBMS? What are the key challenges for implementing
the new visual object computations?
}\bigskip

Thank you for the comment.
In Section 3.2, we have expanded on the crash recovery of PMGD in
comparison to other graph databases.

\bigskip
\textbf{
D2: The coverage of persistent memory is underwhelming because persistent memory
is never used in the experiments.
What is the key advantage of using persistent memory and how is it
more suitable for visual data? Why not add experiments comparing VDMS with
and without using PMGD?
}\bigskip

Due to page limitations and the use of SSDs instead of persistent memory in the
evaluation, we did not elaborate on advantages of PMGD and related experiments.

\bigskip
\textbf{
D3: The experiments section seems unnecessarily long, yet missing some important
details. All the experiments are performed on images and do not handle videos or feature
values at all. For example, video operations may include interval operations or video
clip retrieval. Some parts seem redundant, e.g., the description of Figure 9 seems
near-identical to that of Figure 8.
}\bigskip

The paper is focused on an image search application so we did not include any
experiments on videos or feature vectors. We have experiments that we can provide in
appendix but due to page limitations, it did not fit the scope of this paper.

Figure 8 and 9 are quite similar but provide the performance difference with different
numer of concurrent clients. We have removed figure 9 and replaced with a figure
demonstrating the difference between queries with and without the resize operation.

\bigskip
\textbf{
D4: Fixed all typos
Page 1: “need for multiple system” -> “need for multiple systems”
Page 5: “desing specifically” -> “designed specifically”
Page 7: “images, taglist, and taglist” -> “images, taglist, and autotags”
Page 10, Figure 6: “10M database).” -> “10M database”
}\bigskip

All typos are fixed in the final version.

\pagenumbering{arabic}


\end{document}
