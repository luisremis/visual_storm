\section{Introduction}
\label{intro}

Visual computing workloads performing analytics on images and/or videos
have become prolific across a wide range of application domains.
This is in part due to the growing ability of machine learning (ML) techniques
to extract information from the visual data which can subsequently be used
for informed decision making \cite{vdms-nips}.
The insights this information can provide depend on the
application: retail vendors might be interested in knowing which are the most
visited areas of their stores using security video feeds as input,
or a doctor might want know the effect of a specific treatment by
looking at the changes in size of a tumor from a brain scan.

Despite the increasing use of visual data processing,
there has been very little research on the \textit{management} of visual data.
Most of the current storage solutions for visual data are an ad-hoc collection
of tools and systems, that are re-purposed and adapted to work with visual data.
The approach of re-purposing and integrating solutions not designed for a task
results in resource utilization inefficiencies\cite{rasdaman}.
The combination of systems is required to manage both \textit{metadata} and
\textit{visual data}.
We use \textit{visual data} to refer to any pixel-data (images, videos,
frames in a video, etc), and feature vectors (a.k.a. descriptors, or embeddings),
which are representations of the pixel data.
We use \textit{metadata} to refer to any data that is important within
the context of specific applications.
Information about a patient (name, last name, unique identifiers), or
about a clinic (name, location, etc) are examples of what we refer to
as \textit{metadata} in the context of a health-care IT system.
One can think of \textit{metadata} as the data an application would store
in rows and tables following the relational model.

To illustrate the point on the need for multiple system to build the data
infrastructure for applications,
consider a ML developer constructing a pipeline for extracting
brain tumor information from existing brain images in a classic medical
imaging use case.
This requires assigning consistent identifiers for the scans and adding their
\textit{metadata} in a relational or key-value database\cite{kumar2017data}.
If the queries require a search over patient information,
then patients are associated with their brain scans.
Finally, if the ML pipeline needs images with a different resolution
than the original, there is additional compute
diverted towards pre-processing the original images.
All these steps require understanding different software
solutions that provide various functionalities that can then be stitched
together with many scripts for this specific use case.
Moreover, if the pipeline identifies new metadata to be added for
the tumor images, most databases make it hard to change the schema on the fly.
As another example, many applications can be studied through the use of large
and publicly available datasets.
Applications include basic image search functionality (through the use
of human-generated tags), advanced image search through the use of
machine-generated tags and feature vectors\cite{imagesearch, qin2020similarity}
for each image, and video summarization.
For these use-cases, the usual first step consists on selecting a
subset of the data before running any processing, and a large effort
is devoted to filtering, curating, and pre-processing the data.
Selecting subsets of data is by itself a time consuming task,
as it involves loading all metadata into a solution that enables searching
based on tags (relational database, graph database, csv files, etc), and
building the necessary pipelines for querying and retrieving the right data.

More generally, data scientists and machine learning developers
usually end up building an ad-hoc solution that results in a
combination of databases and file servers to store
metadata and visual data (images, videos), respectively \cite{sculley2015hidden}.
This is integrated with a set of custom scripts that tie multiple systems together,
unique not only to a specific application/discipline but often
to individual researchers or development teams
\cite{mayer2020scalable, sculley2015hidden}.
These ad-hoc solutions make replicating experiments difficult,
and more importantly, they do not scale well when deployed in the real-world.
The reason behind such complexity is \textit{the lack of a system
that can be used to store and access all the data the application needs,
including metadata, images, videos, and feature vectors}.

In this paper, we describe VDMS and show how it provides a
comprehensive solution to the data management for applications
that heavily rely on visual data.
VDMS is an Open Source project designed to enable
efficient access of visual data.
To the best of our knowledge, a rich set of functionalities
designed for visual data management, provided behind an integrated API,
is unique to VDMS and we were unable to find a system with similar functionality.
While there are a number of big-data frameworks~\cite{spark, hadoop}, systems
that can be used to store metadata~\cite{memsql, vertica, mysql, postgresql},
and systems that manipulate a specific category of
visual data~\cite{scidb, rasdaman},
VDMS can be distinguished from them on the following aspects:

\begin{itemize}
\item {\em Design for analytics and machine learning}: By targeting
visual data for use cases that require manipulation
of visual information and associated metadata,
\item {\em Ease-of-use}: By defining a common API that allows applications to
combine their complex metadata searches with operations on resulting visual
data, and together with full support for feature vectors. VDMS goes beyond the
traditional SQL or OpenCV level interfaces that do one or the other.
\item {\em Performance}: We show how a unified system such as VDMS can
outperform an ad-hoc system constructed with well-known discrete components.
Because of the capabilities we have built into VDMS, it handles complex
queries significantly better than the ad-hoc system without compromising the
performance of simple queries.
\end{itemize}

In order to evaluate VDMS in a realistic use case,
we use the YFCC100M dataset\cite{Thomee_2016}.
The YFCC100M dataset is the largest public multimedia collection.
It contains the metadata of around 99.2 million photos
and 0.8 million videos from Flickr,
plus expansion packs \cite{features} that include a variety of multidimensional
data, all of which were shared under one of the various Creative Commons licenses.
We have used this dataset for multiple proof of concepts and
applications within our research lab.

Overall, this paper makes the following contributions:

\begin{itemize}
\item We present the design and implementation of the Visual Data Management System.
To the best of our knowledge, there are no other system capable of
managing visual data and its metadata behind a unified API
and that, at the same time, enable applications to operate over visual data
(images, videos, feature vectors).

\item We introduce the VDMS API in detail, designed for Machine Learning and visual
data analytics at scale.

\item We use our internal use-cases and implementations of visual data processing
pipeline to make a comprehensive evaluation between common ad-hoc solutions
that rely on multiple systems and VDMS.

\item We use the largest public available image dataset, YFCC100m, together
with machine generated tags for our real-world use case, providing a performance
evaluation both from the point of view of high concurrency as well as
scale (the database size, including images and metadata is 12TB in size).
\end{itemize}

\section{Related Work}

There are several databases and data management systems that focus on enabling
analytics or combining transactional and analytic workloads over large
scale data, such as SciDB~\cite{brown2010overview},
BigTable~\cite{chang2008bigtable}, Shark~\cite{xin2013shark},
and Vertica~\cite{lamb2012vertica}.
While these systems do not focus on visual processing as a primary entity,
they are valuable resources in distributing and analyzing very large scale data.
A more concrete application involving visual data include
the Facebook architecture for photos that combine Tao~\cite{tao},
Haystack~\cite{beaver2010finding}, and f4~\cite{muralidhar2014f4} for metadata,
hot/recent data, and warm data respectively.
While the social aspect of Facebook is logically suited to a graph,
the bulk of their data is already stored in MySQL~\cite{mysql},
so they chose to develop a graph-aware cache.
A similar approach was taken by Grail~\cite{fan2015case},
which argued that a syntactic layer written on top of a relational database
could answer graph queries.
Diamond~\cite{huston2004diamond,satyanarayanan2010unique} exploited active
disks to perform discard-based searching order to shrink the amount of data
returned to a reasonable size.
We accomplish the same goal by exploiting persistent memory to store a
graph database for metadata.
