\section{Conclusion}
In this paper, we described VDMS design and implementation and 
show a comprehensive evaluation on our Image Search Application. 
We use one of the largest publicly available datasets: 
The Yahoo Flickr Creative Commons 100M (YFCC100M), 
together with the expansions packs that include 
machine-generated labels and feature vectors.
We show how VDMS compares against a combination of 
industry standard systems, all of which are needed to 
replicate only a portion of VDMS' functionality. 
We see improvements up to 364x in certain queries, 
and an average improvement of about 85x when compared to PostgreSQL.
When compared to MySQL, we see up to 96x speedup
and and average improvement of 31x. 
% Finally, we analyze the different trade-offs of VDMS' 
% descriptors indexes, a functionality 
% that fully integrates with the rest of VDMS interface, 
% providing a powerful and comprehensive API.
The design of VDMS, which was conceived as a 
data management system that treats visual entities 
as first class citizens, can remove inefficiencies 
that result from re-purposing and combining solutions
that were not designed for the job while providing 
simpler and richer interfaces. 
VDMS' easy-to-use interfaces outperform industry standard systems
with a set of functionalities which, to the best of our knowledge,
are not available in any other single data management solution for visual data.
VDMS was designed for analytics and it can efficiently handle complex queries 
which can simplify the design of future applications that rely on visual data.
