% THIS IS AN EXAMPLE DOCUMENT FOR VLDB 2012
% based on ACM SIGPROC-SP.TEX VERSION 2.7
% Modified by  Gerald Weber <gerald@cs.auckland.ac.nz>
% Removed the requirement to include *bbl file in here. (AhmetSacan, Sep2012)
% Fixed the equation on page 3 to prevent line overflow. (AhmetSacan, Sep2012)

\documentclass{vldb}
\usepackage{graphicx}
\usepackage{balance}  % for  \balance command ON LAST PAGE  (only there!)
\usepackage{hyperref}       % hyperlinks
\usepackage{graphicx}
\usepackage{comment}
\usepackage{multicol}
\usepackage{framed}
\usepackage{subcaption}
\usepackage{url}            % simple URL typesetting
\usepackage{booktabs}       % professional-quality tables
\usepackage{amsfonts}       % blackboard math symbols
\usepackage{nicefrac}       % compact symbols for 1/2, etc.
\usepackage{microtype}      % microtypography
\usepackage{amsmath}
\usepackage{mathtools}
\usepackage{minted}

% Include information below and uncomment for camera ready
\vldbTitle{Using VDMS to Index and Search 100M Images}
\vldbAuthors{Luis Remis, Chaunt\'e W. Lacewell}
\vldbDOI{https://doi.org/10.14778/xxxxxxx.xxxxxxx}
\vldbVolume{12}
\vldbNumber{xxx}
\vldbYear{2020}

\begin{document}

% ****************** TITLE ****************************************

\title{Using VDMS to Index and Search 100M Images}
% I shortened the title because of spacing looked nice in first
% page and in some of the last pages.
% The abstract already explained what VDMS is, so it
% should be good.

% possible, but not really needed or used for PVLDB:
%\subtitle{[Extended Abstract]
%\titlenote{A full version of this paper is available
% as\textit{Author's Guide to Preparing ACM SIG Proceedings Using
% \LaTeX$2_\epsilon$\ and BibTeX} at \texttt{www.acm.org/eaddress.htm}}}

% ****************** AUTHORS **************************************

% You need the command \numberofauthors to handle the 'placement
% and alignment' of the authors beneath the title.
%
% For aesthetic reasons, we recommend 'three authors at a time'
% i.e. three 'name/affiliation blocks' be placed beneath the title.
%
% NOTE: You are NOT restricted in how many 'rows' of
% "name/affiliations" may appear. We just ask that you restrict
% the number of 'columns' to three.
%
% Because of the available 'opening page real-estate'
% we ask you to refrain from putting more than six authors
% (two rows with three columns) beneath the article title.
% More than six makes the first-page appear very cluttered indeed.
%
% Use the \alignauthor commands to handle the names
% and affiliations for an 'aesthetic maximum' of six authors.
% Add names, affiliations, addresses for
% the seventh etc. author(s) as the argument for the
% \additionalauthors command.
% These 'additional authors' will be output/set for you
% without further effort on your part as the last section in
% the body of your article BEFORE References or any Appendices.

\numberofauthors{2} %  in this sample file, there are a *total*
% of EIGHT authors. SIX appear on the 'first-page' (for formatting
% reasons) and the remaining two appear in the \additionalauthors section.

\author{
% You can go ahead and credit any number of authors here,
% e.g. one 'row of three' or two rows (consisting of one row of three
% and a second row of one, two or three).
%
% The command \alignauthor (no curly braces needed) should
% precede each author name, affiliation/snail-mail address and
% e-mail address. Additionally, tag each line of
% affiliation/address with \affaddr, and tag the
% e-mail address with \email.
%
% Authors
\alignauthor
Luis Remis\titlenote{Luis was a Research Scientist at Intel Labs until late 2019. Most of this work was done while at Intel Labs.}\\
%       \affaddr{Institute for Clarity in Documentation}\\
%       \affaddr{1932 Wallamaloo Lane}\\
       \affaddr{ApertureData Inc.}\\
       \email{luis@aperturedata.io}
\alignauthor
Chaunt\'e W. Lacewell\\
%       \affaddr{Institute for Clarity in Documentation}\\
%       \affaddr{P.O. Box 1212}\\
       \affaddr{Intel Labs}\\
       \email{chaunte.w.lacewell@intel.com}
}
% There's nothing stopping you putting the seventh, eighth, etc.
% author on the opening page (as the 'third row') but we ask,
% for aesthetic reasons that you place these 'additional authors'
% in the \additional authors block, viz.
%\additionalauthors{Additional authors: John Smith (The Th{\o}rv\"{a}ld Group, {\texttt{jsmith@affiliation.org}}), Julius P.~Kumquat
%(The \raggedright{Kumquat} Consortium, {\small \texttt{jpkumquat@consortium.net}}), and Ahmet Sacan (Drexel University, {\small \texttt{ahmetdevel@gmail.com}})}
%\date{30 July 1999}
% Just remember to make sure that the TOTAL number of authors
% is the number that will appear on the first page PLUS the
% number that will appear in the \additionalauthors section.

\maketitle

%-------------------------------------------------------------------------------
\begin{abstract}
%-------------------------------------------------------------------------------
Data scientists spend most of their time dealing with data preparation,
rather than doing what they know best:
build machine learning models and algorithms to solve previously unsolvable problems.
In this paper, we describe the Visual Data Management System (VDMS),
and demonstrate how it can be used to simplify the data preparation process
and consequently gain in efficiency simply because
we are using a system designed for the job.
To demonstrate this, we use one of the largest available
public datasets (YFCC100M),
with 99+ million images and videos, plus add-ons that include
machine-generated tags and 4K dimensional feature vectors,
for a total of $\sim$13TB of data.
VDMS differs from existing data management systems
due to its focus on supporting machine learning and
data analytics pipelines that rely on images, videos, and feature vectors,
treating these as first class citizens.
We demonstrate how VDMS outperforms well-known and widely used
systems for data management by up to $\sim$35x, with
an average improvement of about 15x for our use-cases, and particularly at scale.
At the same time, VDMS simplifies the process of data preparation and data access,
and provides functionalities non-existent in alternative options.
\end{abstract}

\section{Introduction}
\label{intro}

Visual computing workloads performing analytics on
video or image data, either off-line or streaming,
have become prolific across a wide range of application domains.
This is in part due to the growing ability of machine learning (ML) techniques to
extract information from the visual data which can subsequently be used
for informed decision making \cite{vdms-nips}.
The insights this information can provide depend on the
application: a retail vendor might be interested in the amount of time
want to see the effect of a specific treatment on the size of a tumor.

Despite this rich and varied usage environment, there has been very little
research on the management of visual data.
Most of the current storage solutions for visual data are
an ad-hoc collection of tools and systems. 
For example, consider a ML developer constructing a pipeline
for extracting brain tumor information from existing brain images in a
classic medical imaging use case. 
This requires assigning consistent
identifiers for the scans and adding their metadata in
some form of relational or key-value database. 
If the queries require a search over some patient information, 
then patients have to be associated with their brain scans. 
Finally, if the ML pipeline needs images that
are of a size different than the stored ones, there is additional compute
diverted towards pre-processing after the potentially larger images are fetched. 
All these steps require investigation of different software
solutions that provide various functionalities that can then be stitched
together with a script for this specific use case.
Moreover, if the pipeline identifies
new metadata to be added for the tumor images, most databases make it
hard to evolve the schema on the fly.
As another example, many applications that can be studied through the use of large
and publicly available datasets. 
Applications include basic image search functionality (through the use
of human-generated tags), advanced image search through the use of
machine-generated tags and feature vectors\cite{imagesearch} 
for each image, and video summarization.
For these use-cases, the usual first step consists on selecting a 
subset of the data before running any processing, and a large effort 
is devoted to cleaning up and pre-processing the data.
Selecting subsets of data is by itself a time consuming task,
as it involves loading all metadata into a solution that enables searching
based on tags (relational database, graph database, csv files, etc), and
building the necessary pipelines for querying and retrieving the right data.

More generally, data scientists and machine learning developers 
usually end up building an ad-hoc solution that results in a 
combination of databases and file systems to store 
metadata and visual data (images, videos), respectively. 
This is integrated with a set of custom scripts that tie multiple systems together, 
unique not only to a specific application/discipline but often to individual researchers.
These ad-hoc solutions make replicating experiments difficult, 
and more importantly, they do not scale well when deployed in the real-world.
The reason behind such complexity is the lack of a one-system 
that can be used to store and access all the data the application needs.

In this paper, we show how VDMS~\cite{vdms-nips} provides a comprehensive solutions 
to the data management for applications that heavily rely on visual data. 
VDMS is an Open Source project designed to enable efficient access of visual data.
We also expand on the video and feature vector capabilities of
VDMS, which are part of the latest additions to the system.
We analyze different functionalities and trade-offs for this type of data,
in combination with metadata filtering. 
To the best of our knowledge, this set of functionalities, 
provided behind an integrated API, are unique to VDMS and 
we were unable to find a system with similar functionality.
We show how VDMS can be used as the single and centralize point for data
management and data access even when having multiple modalities of data:
Metadata, Image, Videos, and Feature Vectors.

For this work, we use the YFCC100M dataset\cite{Thomee_2016}. 
The YFCC100M is the largest publicly multimedia collection. 
It contains the metadata of around 99.2 million photos 
and 0.8 million videos from Flickr,
plus expansion packs that include a variety of multidimensional data,
all of which were shared under one of the various Creative Commons licenses.
We have used this dataset
for multiple proof of concepts and applications within our research lab. 

\section{VDMS Design \& Implementation}
\label{arch}

In this section, we briefly describe VDMS design principles and implementation, which is already covered in previous work \cite{vdms-nips}.
Figure \ref{fig:arch} depicts the high-level architecture of VDMS.
VDMS implements a client-server architecture that handles client
requests concurrently, and coordinates query execution across
the metadata and data components in order to return a unified response.

The metadata component is the \textit{Persistent Memory Graph
Database} (PMGD) and the (visual) data component is our Visual Compute Module.
The Visual Compute Module enables machine-friendly enhancements to
visual data, exposing high-level abstractions to the \textit{Request Server}
for dealing with a variety of images and video formats (through OpenCV),
and different methods for indexing for feature vectors
(including Facebook's Faiss \cite{faiss}, TileDB \cite{TileDB}).
VDMS and its components are fully available open source
\footnote{https://github.com/IntelLabs/\{vdms, pmgd\}}.
We briefly describe each of the main components as follows:

\begin{figure}
\centering
\includegraphics[width=1\columnwidth]{figures/vdms_arch.pdf}
\caption{VDMS Architecture}
\label{fig:arch}
\end{figure}

\textbf{Persistent Memory Graph Database:}
We use PMGD to provide an efficient storage
solution addressing the increasing popularity of connected data and
applications that benefit from graph-like processing. We have designed
and implemented an in-persistent-memory graph database, PMGD, optimized
to run on a platform equipped with persistent memory.
PMGD provides a property graph model of data storage with the traditional
atomicity, consistency, isolation, and durability properties expected from
databases. The graph model makes it very suitable for the data model and
access patterns shown by visual metadata.
With its natural ability to extend the schema very
easily (due to the use of a property graph model),
we can support new developments in machine learning that can lead to
enhancements to existing metadata over time.
PMGD is designed and optimized for persistent memory technologies
like Intel Optane~\cite{IntelXPoint15}, which
promise storage providing nearly the speed of DRAM and the
durability of block-oriented storage.

\textbf{Visual Compute Module:} This module was designed to provide
an abstraction layer for interacting with visual data.
For traditional formats (jpg, png, tiff, mp4, etc.),
the interface is an abstraction layer over OpenCV. However, it also provides a
way to use novel formats that are better suited for visual analytics: a novel,
array-based lossless image format. This format is built on the array data
manager TileDB~\cite{TileDB} and is well suited for images that are used in
visual analytics.
This module also provides support for videos, enabling operations like
encoding/decoding/transcoding and resize as part of the VDMS interface.
Feature vector support is provided through an implementation based
on high-dimensional sparse arrays, also using TileDB.
In addition, the Visual Compute Library provides a wrapper
for another high-dimensional index implementation,
Facebook's Faiss~\cite{faiss}.

\textbf{Request Server:}
Developers and users of machine learning frameworks and data science
applications favor simpler interfaces to access and process data. They cannot be expected to deal with two different ways of interacting with information
(metadata and visual data) instead of focusing on the
algorithmic parts of their pipelines.
VDMS takes care of coordinating client requests across the metadata and the
data, and managing multiple clients through its Request
Server component by implementing a JSON-based API.
It decomposes the command into
metadata and data requests, invokes the relevant calls behind the scene,
and returns a coherent response to the user after applying any additional
operations.

\textbf{Client Library:}
A user application can use the VDMS API by defining metadata conforming to the
query protocol we have defined.
The client side of the VDMS library provides a simple query function that
accepts a JSON string with commands and an array or vector of blobs.
Internally, the library wraps the query string and blob using
Google Protobufs \cite{protobufs} and sends it to the VDMS server.
It also receives a similarly formed response from VDMS
and returns it to the client. The responses require JSON parsing on the client
side for the metadata string that indicates how to interpret the blobs field.

\textbf{VDMS API:}
VDMS API is easy to use and explicitly pre-defines certain
visual primitives associated with metadata, images, videos, and feature vectors.
While we use a graph database to store our metadata,
the API is not graph-specific.
Authors have paid particular attention to hide the complexities of our internal
implementation and up-level the API to a JSON-based
API\footnote{https://github.com/IntelLabs/vdms/wiki/API-Description},
which is very popular across various application domains.
By defining a new JSON-based API, there is a trade-off between
expressiveness (compared to SPARQL or Gremlim, or even SQL) and
the ability to natively support visual data operations.
However, it should be possible to achieve similar levels of
expressiveness compared to more mature query languages over time.
The current front-ends available are a Python and C++ client
library to provide a simple query function that accepts a JSON string with
commands and an array (or vector) of blobs.

While there are a number of big-data frameworks~\cite{spark, hadoop}, systems
that can be used to store metadata~\cite{memsql, vertica}, and systems that
manipulate a specific category of visual data~\cite{scidb, rasdaman}, VDMS can
be distinguished from them on the following aspects:

\begin{itemize}
\item {\em Design for analytics and machine learning}: By targeting
visual data for use cases that require manipulation
of visual information and associated metadata,
\item {\em Ease-of-use}: By defining a common API that allows applications to
combine their complex metadata searches with operations on resulting visual
data, and together with full support for feature vectors. VDMS goes beyond the
traditional SQL or OpenCV level interfaces that do one or the other.
\item {\em Performance}: We show how a unified system such as VDMS can
outperform an ad-hoc system constructed with well-known discrete components.
Because of the capabilities we have built into VDMS, it handles complex
queries significantly better than the ad-hoc system without compromising the
performance of simple queries.
\end{itemize}

\section{Evaluation}
\label{eval}

\subsection{YFCC100M Dataset}
\label{dataset}

The Yahoo! Flickr Creative Commons 100m (YFCC100M) dataset is a large collection of 100 million public Flickr media objects created to provide free, sharable multimedia data for research. This dataset contains approximately 99.2 million images and 0.8 million videos with metadata characterized by 25 fields such as the identifier, user, date the media was taken/uploaded, location in longitude/latitude coordinates, device the media was captured, URL to download the media object, and Creative Commons license information.  The YFCC100M dataset also used a deep learning approach to generate autotags which is a set of comma-separated concepts such as people, scenery, objects, and animals and confidence scores generated from 1,570 trained Caffe classifiers ~\cite{Thomee_2016}.
We have also used feature vectors generated for every image and first frame
of every video \cite{features} to implement similarity search.

\subsection{Experimental Setup}

For all our experiments, we use 2 servers, one hosting VDMS server and
another hosting MySQL server. Both servers have a dual-socket
Intel\textsuperscript{\textregistered}
Xeon\textsuperscript{\textregistered} Platinum 8180 CPU @ 2.50GHz (Skylake),
each CPU with 28 physical cores with multithreading enabeled,
for a total of 112 logical cores per server.
The server hosting MySQL has 256GB of DDR4 DRAM, while the server hosting VDMS
has 64GB of DDR4 DRAM. Both servers run Ubuntu 16.04.
The servers and client are connected through a 1GB wired link.

As a baseline, we implemented a similar database system comprised of a
combination of widely available and out-of-the-box components: MySQL Server 5.7,
Apache Web Server 2.4.18, and OpenCV 3.3.
Figure~\ref{fig:systems} shows a logical view of the difference between the
interaction of the client application (which wants to retrieve metadata and
imaged) with VDMS (left) and the baseline (right).
It is worth noting that the images are stored in a shared repository
(ext4 filesystem on a RAID 6 configuration of 16TB) that both Apache WebServer
and VDMS have direct access. In the case of MySQL, metadata is stored in an
attached SSD disk. Even if VDMS has native support for Optane Persistent Memory,
we do not use it in this experiment because of fairness of comparison w.r.t
MySQL. The benefits of Persistent Memory on metadata operations is left
for another paper, and outside the scope of this evaluation.
For this experiment then, we simply use a similar attached SSD disk to store
metadata (even if VDMS Graph store is designed for PM, it can deliver good
performance when using SSDs directly).

\begin{figure*}[]
\centering
\includegraphics[width=\textwidth]{figures/comparison_system}
\caption{Comparison Systems}
\label{fig:systems}
\end{figure*}

We built VDMS and MySQL databases using the YFCC100M dataset with incremental database sizes 100k, 500k, 1M, 5M, 10M, 50M, and 100M. For each database size, we created a PMGD graph from the YFCC100m metadata, the YFCC100m autotags associated with each metadata identifier, and a list of 1,570 autotags using 100 concurrent VDMS Python clients.  With simple queries, we insert 1,570 nodes for the autotags and insert nodes for the YFCC media objects with the appropriate metadata.  Adding the autotag confidence scores is a recursive process and requires more complex queries than adding the previous nodes. For each metadata identifier in autotags, the query must find the associated metadata node and tag node.  A connection between the two nodes is created with the confidence score as a property. The number of metadata nodes are dependent on the database size and the connections are responsible for 90\% of the elements in each database as shown in Table~\ref{table:vdmsnodes}. It is important to note that the metadata identifier, autotags, and longitude/latitude coordinates are set as indexes of the database to allow faster retrieval of the metadata.

\begin{table}[h]
\caption{Nodes in VDMS database}
\centering
\begin{tabular}{c c c c}
\hline\hline
Database Size & Connections & Images & Autotags\\
\hline
100k & 848,432      & 100,000     & 1,570\\
500k & 4,249,500    & 500,000     & 1,570\\
1M   & 8,503,045    & 1,000,000   & 1,570\\
5M   & 42,505,478   & 5,000,000   & 1,570\\
10M  & 85,040,404   & 10,000,000  & 1,570\\
50M  & 425,162,070  & 50,000,000  & 1,570\\
100M & 895,572,430  & 99,205,984  & 1,570\\
\hline
\end{tabular}
\label{table:vdmsnodes}
\end{table}

Each MySQL database is created in a similar manner but the data is represented as three tables: taglist, metadata, and autotags.  By default, MySQL uses inifinite number of threads inside InnoDB when processing requests using four threads per IO read/write.  In some cases, when creating large databases data locks may occur to protect the data from concurrent updates.  To minimize locking in our runs, we increased the InnoDB buffer pool size to increase the amount of memory allocated to in-memory data structures ~\cite{mysql,mysql_blog}. Using a Python client and simple queries, the taglist table is read from the list of tags with an auto-incremented tagid as an index and the metadata table is read from the YFCC100m metadata using the identifier as an index. The autotags table contains the generated autotags and confidence scores for metadata entries of the metadata table. To generate the table, we split the autotags data for each database by the metadata identifier and autotag into new files. The new files are read into the respective autotags table with the metadata identifier and tagid as indexes.

\begin{table}[h]
\caption{Rows in MySQL database}
\centering
\begin{tabular}{c c c c}
\hline\hline
 & \multicolumn{3}{c}{Table}\\
\cline{2-4}
Database Size & Autotags & MetaData & Tag List\\
\hline
100k & 848,912     & 100,000    & 1,570\\
500k & 4,241,200   & 498,707    & 1,570\\
1M   & 8,508,380   & 1,000,000  & 1,570\\
5M   & 42,425,905  & 4,987,379  & 1,570\\
10M  & 85,095,265  & 10,000,000 & 1,570\\
50M  & 425,446,208 & 50,000,000 & 1,570\\
100M & 896,002,496 & 99,206,564 & 1,570\\
\hline
\end{tabular}
\label{table:mysqltables}
\end{table}

\begin{figure}[]
\centering
\includegraphics[width=\columnwidth]{figures/db_time_size}
\caption{Time (in hours) to build and size (in GB) of MySQL and VDMS databases}
\label{fig:db_time_size}
\end{figure}

The VDMS and MySQL databases have comparable number of elements as shown in Table~\ref{table:vdmsnodes} and ~\ref{table:mysqltables} but VDMS outperforms MySQL in build speed.  Figure~\ref{fig:db_time_size} illustrates how VDMS can build databases faster than MySQL as the database size grows.  Key difference in the build times are contributed to the low-level implementation of how MySQL reads and stores data from the files and the optimizations (increased InnoDB pool size, etc.) needed to handle large datasets such as YFCC100m.  On average, it took MySQL 3.72x hours longer to build each database than VDMS. For example, to build the 100M database, MySQL took 263 hours while VDMS only needed 72.5 hours.  This is a difference of over 7 days of processing data with VDMS missing less than 0.001\% images and 0.048\% connections from the entire dataset.  Alternatively, VDMS requires more storage, shown in Figure~\ref{fig:db_time_size}, to store information about each node/connection.  This may become a factor if storage is a limitation.  For our runs, VDMS required 30-41\% more storage than MySQL.


\subsection{Images + Metadata}

To evaluate the access to metadata and images, we use the following four queries:
\begin{enumerate}
\item {\bf {\em 1tag}}: Find metadata/images of alligator.
\item {\bf {\em 1tag\_resize}}: Find metadata/images of alligator and resize to 224x224.
\item {\bf {\em 1tag\_resize\_geo}}: Find metadata/images of alligator, resize to 224x224, and in a particular geolocation (20 degrees radius).
\item {\bf {\em 2tag\_resize}}: Find metadata/images of BOTH alligator and lake and resize to 224x224.
\end{enumerate}

\begin{figure}[]
\centering
\includegraphics[width=\columnwidth]{figures/concurrency_comparison}
\caption{Concurrency Analysis - VDMS vs Baseline}
\label{fig:concurrency_vdms}
\end{figure}

\begin{figure}[t!]
\centering
\includegraphics[width=\columnwidth]{figures/queries_throughput_32}
\caption{Queries - Throughput - 32 Clients}
\label{fig:q_throughput_32}
\end{figure}

\begin{figure}[t!]
\centering
\includegraphics[width=\columnwidth]{figures/queries_throughput_56}
\caption{Queries - Throughput - 56 Clients}
\label{fig:q_throughput_56}
\end{figure}

An image and metadata search on the YFCC100M databases are very extensive
and would benefit from multithreading to process queries simultaneously.
Figure~\ref{fig:concurrency_vdms} illustrates a concurrency analysis for
the \textit{1tag} query using VDMS to investigate the number of
concurrent clients to use for the per-query performance analysis.
This figure shows the metadata performance plateaus at 32 clients and
the image performance at 56 clients for a simple query.
To further investigate, Figure~\ref{fig:q_throughput_32} and
Figure~\ref{fig:q_throughput_56} illustrate the throughput performance
for the \textit{1tag} and \textit{1tag\_resize} queries for 32 and 56 clients,
respectively.
On the 56-client experiment, a 10x difference occurs between VDMS and
MySQL in metadata throughput while 32 clients is similar but not as high.
The overall performance for both MySQL and VDMS is better with 56 clients;
therefore, we used this value for our per-query analysis.

\begin{figure}[]
\centering
\includegraphics[width=\columnwidth]{figures/q1_latency}
\caption{Query 1 - Latency}
\label{fig:q1_latency}
\end{figure}

\subsection{Videos}

\begin{figure*}[ht!]
\centering
\includegraphics[width=\textwidth]{figures/video_overhead}
\caption{Concurrency (left) and Overhead (right)}
\label{fig:video}
\end{figure*}

VDMS provides full support for video storage and operations.
This includes support for encoding/decoding and trans-coding of mp4, avi, and
mov containers, as well as support for xvid, H.263 and H.264 encoders.
This is supported through the Visual Compute Module that provides an abstraction
layer on top of OpenCV (and ffmpeg directly).
All the operations supported for images in VDMS are also supported at the
video and frame level on the API. On top of that, there are a number of
video-specific operations that are supported, such as the interval operations,
that allow users to retrieve clips at different FPS versions of the video.

All this functionality is provided and integrated with the rest of the
metadata API as part of the comprehensive VDMS interface. This makes it
possible for users to interact with metadata and video in a transactional
manner, enabling users to run queries like: "Retrieve all the videos
where there is a lake with probability higher than X, converting all videos
to H.264 mp4 of size 224x244".
In particular, this functionality was used internally to select a subset
of videos with the right licenses for a video summarization application.

To the best of our knowledge, there is no solution that can provide
all the functionality mentioned above. Also, implementing a baseline is a very
complex task as there is a large number of options and parameters that can
be chosen, which makes it hard to accurately compare against VDMS functionality.
For this reason, we chose to make a study using VDMS in various scenarios,
and analysis what is the impact of having the overhead of VDMS' Request
Server in the overall access time.

Figure \ref{fig:video} shows the analysis of different queries aimed to retrieve
a video using VDMS interface. We show how VDMS increases throughput of serving
a video object as the number of simultaneous client increases, as well as the
overhead VDMS introduces in the overall query execution time.
The figure on the left compares the number of video transaction per second
(i.e., number of videos returned per second) when different operations
are executed as part of the transaction. The upper-bound of this would be
simply returning the video as-is (without running any encoding/decoding or
operation), represented by the red line. This query is the upper-limit because
it essentially translates to reading the video from the file-system and sending
it over a TCP/IP socket, without any other overhead or operations.

We also run a set of other queries that involve: (a) running a resize operation
on the video and, consequently, needs a decoding and
encoding operations as well (blue line),
(b) transcoding, meaning the use a different container and encoder
than the one originally used (yellow line), and (c) both resize and transcoding.
Note that the resize operation performs a downsize, which translates in less
data being sent over the wire. This is specially noticeable when supporting 32
simultaneous clients, where the systems provides more videos per second due to
sending less data to the client, when compares to just transcoding and not resizing (yellow line).

We can see that the system performs best when using all the physical cores.
COMPLETE THIS BETTER.

WE SHOULD ADD HERE SAMPLES OF THE QUERYS TO SHOW HOW THESE ARE RETRIEVED.

Because we see almost 3 orders of magnitude drop in performance when including
operations as part of the query, we wanted to understand where most of the time
was spent on the query, and optimize the Request Server and Visual Compute Module
if necessary. For this, we run the experiment shown at
Figure \ref{fig:video} (right) which breaks down the different components of the
queries. This figure shows that more than 97\% of the query execution is spent
on encoding/decoding operations, which is well-known to be a
heavy and compute intensive operation.
On the one hand, this results show that VDMS introduced overhead for
video operation is minimal. On the other hand, this result means a
limit on the opportunities for optimization for video queries given
that biggest time factors are accounted by encoding/decoding, which is
outside the scope of VDMS.
This result also was the inspiration point for one optimization we included
in future versions of VDMS, which involves using ffmpeg C++ API to
limit the number of frames being encoding/decoding when possible.


\subsection{Feature Vectors}

Another key differentiating factor of VDMS is that it allows the creation of
indexes for high-dimensional feature vectors and the insertion of
these feature vectors associated with entities or visual objects.
Feature vectors are intermediate results of various machine
learning or computer vision algorithms when run on visual data.
Feature vectors are also known as "descriptors" or "visual descriptors".
We use these terms interchangeably.
These descriptors can be labeled and classified to build search indexes,
and there are many in-memory libraries that are designed for
this task~\cite{flann, faiss}.
Using the VDMS API, users can manage feature vector indexes,
query previously inserted elements (images),
run a k-nearest neighbor search (knn) and express relationships
between existing images or descriptors and
the newly inserted descriptors.
By natively supporting descriptors and knn,
VDMS allows out-of-the-box classification
functionalities for many applications.

For this work, and as part of a comprehensive image search implementation,
we have used 4096-dimensional descriptors extracted from every image
(and first frame of every video) and created a collection of these feature
vectors in VDMS to be able to perform similarity search (i.e., find
images that are "similar" to an query (input) image).
"Similarity" in this particular case is defined as closeness
in a an 4096-dimensional space using euclidean distance as the metric.

EXPLAIN HERE DIFFERENT ACCURACY/EXECUTIME TIME TRADEOFF.

The process of loading descriptors in VDMS is simple.
First, the user has to create a DescriptorsSet, using a single command.
At creation of the DescriptorSet, the dimensionality of the descriptors
is specified, together with the desired indexing method and the desired metric
for computing distances (Euclidean Distance, L2, or Inner Product, IP).
Once the DescriptorSet is created, descriptors can be inserted to the set.
After the descriptors are inserted, similarity search can be performed.
Note that descriptors can be inserted later on, and any search will reflect
the existence of all descriptors in the set.

\begin{figure*}[]
\centering
\includegraphics[width=\textwidth]{figures/feature_img_results}
\caption{Sample Results of Similarity Search}
\label{fig:similarity}
\end{figure*}

Figure \ref{fig:similarity} shows 3 examples of a query image (on the left),
and images returned as "similar" by VDMS.
The query input is a descriptor generated after a query image. The "query"
descriptor is sent to VDMS as part of the query, and VDMS use that descriptor
to find similar one, and retrieve the images associated with those "similar"
descriptors. We used this as an example and as a visual validation of the
functionality and applicability in this particular dataset, but we also
provide an analytical approach to accuracy and trade-offs in our system.
It is important to note that the accuracy of the results is entirely tied
to the quality of the descriptors chosen by the applications.
VDMS is completely agnostic to information contained within the descriptor,
and simply offers the interface to store and index them, but the quality
of the similarity result will be tied to the quality of descriptor extraction
that the application is using.


\begin{figure*}[]
\centering
\includegraphics[width=\textwidth]{figures/features_alternatives}
\caption{Feature Vector Evaluation: Trade-off between query execution speed
and accuracy of the results, using ground-truth data for computing accuracy.
For this evaluation, we query the 10 closest neighbors (k = 10), and compute
accuracy using recall at 4 (r\_k = 4) (i.e. percentage of the top 4 ground-truth
results that is present within the top 10 computed neighbors).
We average the query execution time and accuracy for 100 queries (nq = 100).}
\label{fig:features_eval}
\end{figure*}

As mentioned before, VDMS provides different levels of customization of the
indexes created for a descriptor set, that includes the indexing techniques
and the metric for similarity.
These different indexing techniques comes with different trade-offs in terms
of speed of search and accuracy.
VDMS aims to provide functionality that is agnostics to application-specific
techniques, enabling functionality that is general to visual data processing
applications.
Figure \ref{fig:features_eval} shows an analysis at the different indexing
techniques provided by VDMS and its trade-off between accuracy and query
execution speed, for a single threaded client.
For this evaluation, we query the 10 closest neighbors (k = 10), and compute
accuracy using recall at 4 (r\_k = 4) (i.e. percentage of the top 4 ground-truth
results that is present within the top 10 computed neighbors).
We average the query execution time and accuracy for 100 queries (nq = 100).
The "flat" index (red line) implements exact search and
represents ground-truth, which explain why the accuracy is always 100\% on the
right plot. The other indexes implement "approximate search", which trade-offs
between accuracy and speed of search~\cite{flann, faiss}.
We have also tried the "ivfflat" index (inverted file index), as well as
LSH-based indexes using a different number of bits per descriptor.
\footnote{https://github.com/facebookresearch/faiss/wiki/Faiss-indexes}.
Results show how "ivfflat" is the fastest option by comes at the trade-off
of about 30\% lost in accuracy, while simple brute-force search
is among the slowest options at the expenses of 100\% accuracy,
meaning exact search.

\begin{figure*}[]
\centering
\includegraphics[width=\textwidth]{figures/features_disksize}
\caption{Feature Collection Size in Disk}
\label{fig:features_size_does_matter}
\end{figure*}

Another important trade-off to be made is w.r.t to space efficiency: Set of
descriptors can grow very large and hard to manage.
In this particular case, 4096-dimensional descriptors for 100M elements
translates into 1TB of data, only in raw floating-point data alone (without
accounting for any metadata or indexes associated with it).
This component is very important on the overall analysis because a large set
of feature vectors may not fit in memory and thus cause a pressure on the IO
system while retrieving descriptors for computing distance, that severely impact
the overall query execution time.
Because of this, when the set of descriptors grows significantly large,
it may be worth trading off accuracy to speed and space.
Figure \ref{fig:features_size_does_matter} show the different indexes and
their size in disk. These indexes already contain all the descriptors (or
a quantized version of them), and can be loaded in memory directly when it fits.
Note how, because of quantization of the descriptors,
LSH provides a significantly lower space foot print, which can be a great option
for large collections of descriptors when accuracy is not a main factor.


\section{Conclusion}

We introduced the Visual Data Management System,
designed to enable efficient access of visual data.
We presented our rich, JSON-based API, designed to simplify visual data access
for data scientists and analytics pipelines.
We compared the performance of our system to a
baseline of a combination of widely available and used systems.
Our findings showed that VDMS efficiently deals with complex queries,
providing a performance improvement
of up to 2x in the examined medical data search use-case.
Furthermore, VDMS requires significantly fewer
lines of code to execute complex queries in complex visual pipelines.
We intend to continue with the evaluation of
VDMS performance different use cases, and to identify more
opportunities to optimize and simplify the access of visual data
for machine learning and analytics applications.


\section{Acknowledgments}

We would like to thank the many people that made this project possible and
helped us through the process, as this work is the results of many efforts.
We want to specially thank our senior technologists Nilesh Jain and Ravi Iyer for their
full support and input during the duration of the project.
We want to specially acknowledge Philip Lantz and Vishakha Gupta for their help
with PMGD, key to load large datasets into VDMS.
We want to thank Jim Blakley for his input during the various 
phases of our project, and for advocating and promoting its use.
We want to acknowledge the Intel Labs VDMS team for their efforts
open-sourcing and maintaining the system.
We want to thank Jason Gardner for his helping in setting up many of the servers
and infrastructure needed to conduct our experiments.



\clearpage

% The following two commands are all you need in the
% initial runs of your .tex file to
% produce the bibliography for the citations in your paper.
\bibliographystyle{plain}
\bibliography{main}
\input{legal.tex}

\clearpage
\begin{appendix}

\begin{comment}
You can use an appendix for optional proofs or details of your evaluation which are not absolutely 
necessary to the core understanding of your paper.
\end{comment}

\begin{listing}[ht!]
\begin{minted}[frame=single,
               framesep=3mm,
               linenos=true,
               xleftmargin=21pt,
               tabsize=4]{js}
"FindEntity"{     
    "class": "autotag",
    "constraints": { 
        "name": ["==", "alligator"]
    }
    "_ref" : 1
},
"FindImage":{
    "format": "png",
    "link": {
        "ref":1,
        "constraints": {
            "prob": [">=", 0.66]
        }
    }
    "operations": [{
        "type": "resize",
        "height": 224,
        "width":  224,
    }]
}

\end{minted}
\caption{Sample Query for Images - 
The query expresses the following: 
Find all the images connected to the autotag \textit{alligator} 
with probability higher than 0.66, apply a resize operation
to make the images 224x224, and convert to "png".} 
\label{findimage}
\end{listing}

\begin{listing}[ht!]
\begin{minted}[frame=single,
               framesep=3mm,
               linenos=true,
               xleftmargin=21pt,
               tabsize=4]{js}
"FindEntity"{     
    "class": "autotag",
    "constraints": { 
        "name": ["==", "lake"]
    }
    "_ref" : 1
},
"FindVideo":{
    "container": "mp4",
    "codec": "h.264",
    "link": {
        "ref":1,
        "constraints": {
            "prob": [">=", 0.86]
        }
    }
    "operations": [{
        "type": "resize",
        "height": 1080,
        "width":  1920,
    }]
}

\end{minted}
\caption{Sample Query for Video - 
The query expresses the following: 
Find all videos connected to the autotag \textit{lake} 
with probability higher than 0.86, apply a resize operation
to make the video 1920×1080, and convert to "mp4" file, 
using H.264 enconding.} 
\label{findvideo}
\end{listing}

\begin{listing}[t!]
\begin{minted}[frame=single,
               framesep=3mm,
               linenos=true,
               xleftmargin=21pt,
               tabsize=4]{js}
"FindEntity"{     
    "class": "autotag",
    "constraints": { 
        "name": ["==", "alligator"]
    }
    "_ref" : 1
},
"FindImage":{
    "format": "png",
    "link": {
        "ref":1,
        "constraints": {
            "prob": [">=", 0.66]
        }
    }, 
    "constraints": {
        "latitude": [">=", 36.23433, 
                     "<=", 38.23433]
        "longitude":[">=", -114.80666, 
                     "<=", -116.80666]
    },
    "operations": [{
        "type": "resize",
        "height": 224,
        "width":  224,
    }, {
        "type": "rotate",
        "angle": 45.34
    }]
}

\end{minted}
\caption{Sample Query for Images - 
The query expresses the following: 
Find all the images connected to the autotag \textit{alligator} 
with probability higher than 0.66, 
filter by latitude and longitude within 1 degree, 
apply a resize operation to make the images 224x224
and rotate the image 45.34 degrees, 
and return the images as "png" files.} 
\label{findimagegeo}
\end{listing}

\end{appendix}

\begin{appendix}

\begin{comment}
You can use an appendix for optional proofs or details of your evaluation which are not absolutely 
necessary to the core understanding of your paper.
\end{comment}

\begin{listing}[ht!]
\begin{minted}[frame=single,
               framesep=3mm,
               linenos=true,
               xleftmargin=21pt,
               tabsize=4]{js}
"FindEntity"{     
    "class": "autotag",
    "constraints": { 
        "name": ["==", "alligator"]
    }
    "_ref" : 1
},
"FindImage":{
    "format": "png",
    "link": {
        "ref":1,
        "constraints": {
            "prob": [">=", 0.66]
        }
    }
    "operations": [{
        "type": "resize",
        "height": 224,
        "width":  224,
    }]
}

\end{minted}
\caption{Sample Query for Images - 
The query expresses the following: 
Find all the images connected to the autotag \textit{alligator} 
with probability higher than 0.66, apply a resize operation
to make the images 224x224, and convert to "png".} 
\label{findimage}
\end{listing}

\begin{listing}[ht!]
\begin{minted}[frame=single,
               framesep=3mm,
               linenos=true,
               xleftmargin=21pt,
               tabsize=4]{js}
"FindEntity"{     
    "class": "autotag",
    "constraints": { 
        "name": ["==", "lake"]
    }
    "_ref" : 1
},
"FindVideo":{
    "container": "mp4",
    "codec": "h.264",
    "link": {
        "ref":1,
        "constraints": {
            "prob": [">=", 0.86]
        }
    }
    "operations": [{
        "type": "resize",
        "height": 1080,
        "width":  1920,
    }]
}

\end{minted}
\caption{Sample Query for Video - 
The query expresses the following: 
Find all videos connected to the autotag \textit{lake} 
with probability higher than 0.86, apply a resize operation
to make the video 1920×1080, and convert to "mp4" file, 
using H.264 enconding.} 
\label{findvideo}
\end{listing}

\begin{listing}[t!]
\begin{minted}[frame=single,
               framesep=3mm,
               linenos=true,
               xleftmargin=21pt,
               tabsize=4]{js}
"FindEntity"{     
    "class": "autotag",
    "constraints": { 
        "name": ["==", "alligator"]
    }
    "_ref" : 1
},
"FindImage":{
    "format": "png",
    "link": {
        "ref":1,
        "constraints": {
            "prob": [">=", 0.66]
        }
    }, 
    "constraints": {
        "latitude": [">=", 36.23433, 
                     "<=", 38.23433]
        "longitude":[">=", -114.80666, 
                     "<=", -116.80666]
    },
    "operations": [{
        "type": "resize",
        "height": 224,
        "width":  224,
    }, {
        "type": "rotate",
        "angle": 45.34
    }]
}

\end{minted}
\caption{Sample Query for Images - 
The query expresses the following: 
Find all the images connected to the autotag \textit{alligator} 
with probability higher than 0.66, 
filter by latitude and longitude within 1 degree, 
apply a resize operation to make the images 224x224
and rotate the image 45.34 degrees, 
and return the images as "png" files.} 
\label{findimagegeo}
\end{listing}

\end{appendix}


% \section{Final Thoughts on Good Layout}
% Please use readable font sizes in the figures and graphs. Avoid tempering with the correct border values, and the spacing (and format) of both text and captions of the PVLDB format (e.g. captions are bold).

% At the end, please check for an overall pleasant layout, e.g. by ensuring a readable and logical positioning of any floating figures and tables. Please also check for any line overflows, which are only allowed in extraordinary circumstances (such as wide formulas or URLs where a line wrap would be counterintuitive).

% Use the \texttt{balance} package together with a \texttt{\char'134 balance} command at the end of your document to ensure that the last page has balanced (i.e. same length) columns.

\end{document}
